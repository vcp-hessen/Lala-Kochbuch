\newpage
\section{\cancel{Un}nützliches Küchenwissen}
\begin{itemize}
    \item Unsere Sipplinge trinken keinen Tee – jaja das hört man immer wieder, aber was ist mit Eistee? Den liebt ja wohl jedes Kind.

    Einfach einen großen Topf Tee kochen, Holundersirup, Zitronensaft und Minze einrühren und abkühlen lassen. Kalt ein Gedicht, wenn dann noch was da ist – warm ist es nämlich auch klasse.
    Achtung: Holundersirup und Zitronensaft gibt’s nicht von uns, aber beim Sondereinkauf oder einfach von zu Hause.
    
    \item Wenn man das Spülwasser auf kleiner Flamme bereits dann aufsetzt, wenn man zum Essen geht, kann man direkt im Anschluss spülen. Natürlich den Deckel nicht vergessen.
 
    \item Oft essen Vegetarier, Veganer und Omnivoren ja fast dasselbe – bis auf die kleinen, feinen, fleischigen Details. Oft kann man die aber einfach später in den Topf werfen, so dass ihr erstmal für alle das Gleiche kocht, dann aufteilt und perfektioniert. Viel weniger Arbeit.

    \item „Ihhhh Gurken“ hört man es aus dem Nachbarstamm schreien und wünscht sich eben diese, weil im eigenen Essenskreis keine mehr da ist. \emph{Let's swap}, sag ich da nur. Bringt Lebensmittel, die bei euch im Stamm keiner mag zum Tausch/Swap-Regal bei der Verpflegung und nehmt euch mit, was anderen nicht schmeckt. 
Aber Leute: Bitte nur Sachen, die noch im Rohzustand sind, nix gekochtes – damit läuft man über den Platz und bietet es hungrigen Stämmen an.

    \item Tauschen solltet ihr auch unbedingt bei Marmeladen. Von uns bekommt ihr keine, aber Marmelade gibt’s für alle – von euch daheim in Liebe gekocht und mit aufs Lager gebracht. Übrigens auch 'n super Tip, um endlich den Schwarm aus dem Nachbarstamm kennenzulernen: frag ihn doch einfach, ob sie oder er deine Marmelade probieren möchte.
 
    \item Für die richtigen Klugscheißer unter uns – wusstet ihr eigentlich, dass ...
    \begin{itemize}
        \item ...Brot länger frisch bleibt, wenn es am Stück ist?
        \item ...Salat und Dressing am besten getrennt serviert werden sollten, damit man beides noch aufheben kann?
        \item ...Wasser schneller kocht, wenn der Deckel drauf ist?
        \item ...man mit Reis und einem Topf zaubern kann (der Trick ist im Heft versteckt)?
        \item ...man sich eigene Kühlschränke bauen kann?
        \item ...man Wasser besser erst salzen sollte, wenn es kocht?
        \item ...Rollgerste 1000[..]00 mal besser ist als Reis (nähere Infos bei Jo aus Grävenwiesbach)?
        \item ...Kinder alles essen, wenn man ihm coole Namen gibt? \emph{„Das sind keine Krautnudeln – das sind Kraftspätzle mit magischem Kraut“}
        \item ...es gaaaaanz, gaaaanz viele Tschairezepte gibt und alle das anders machen?
        \item ...
    \end{itemize}
\end{itemize}