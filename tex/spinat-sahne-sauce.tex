%!TEX root = ../Kochbuch.tex

% Complete recipe example
\begin{recipe}{Spinat-Sahne-Sauce mit Nudeln}
    \graph{
        small=pic/spinat-sahne-sauce,
        big=pic/spinat
    }
    
    % \introduction{%
    % Risotto ist DAS Gericht, das das Prädikat "{}schlotzig"{} trägt. Gemeint ist die Konsistenz, die der Reis erreichen soll und die ist genauso, wie es das Wort vermuten lässt.
    % }
    
    \ingredients{%
       1,3 kg & Nudeln \emph{(Vollkorn)}\\
       1,2 kg & Spinat\\
       500 g & Sahne\\
       500 g & Zwiebeln\\
       1 Knollen & Knoblauch
    }
    
    \preparation{%
        \step Zunächst werden die Nüdelchen aufgesetzt und nach Anleitung gekocht... Ihr kennt das ja - Wasser kochen, Salz dazu und dann die Nudeln dazu - ab und zu umrühren. 
        
        \step Den frischen Spinat müsst ihr vor dem Essen Waschen und dann blanchieren (sprich: blongschieren). Das bedeutet, Ihr bringt leicht gesalzenes Wasser zum Kochen und legt die Spinatblätter, die Ihr schon klein geschnitten habt, in das kochende Wasser. 
        
        \step Nach einer halben Minute fischt Ihr die Blätter wieder heraus und steckt sie zum Abkühlen in einen anderen Topf mit kaltem Wasser. Der Spinat wird sehr viel Volumen einbüßen, also habt Ihr am Ende viel weniger Kram, der in die Sauce kommt.
        
        \step Währenddessen könnt Ihr schon Zwiebeln und Knoblauch klein schneiden, salzen und in Öl anschwitzen. 
        
        \step Wenn die Zwiebeln schön glasig sind, kann der Spinat Einzug in den Kochtopf erhalten. Lasst das Ganze ca. 5 Minuten köcheln und gebt dann Sahne und Gemüsebrühe dazu. 
        
        \step Das ganze kann jetzt mit Salz, Pfeffer und Muskat abgeschmeckt werden. 
        
    }
    
    \setHeadlines{hinthead = Echt wahr!}
    \hint{%
        Manche Stimmen bezeichnen dieses Gericht als das beste Abendessen, das es heute gibt.
    }
    
\end{recipe}