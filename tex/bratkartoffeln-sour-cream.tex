%!TEX root = ../Kochbuch.tex

% Complete recipe example
\begin{recipe}{Bratkartoffeln mit Sour-Cream}
    \graph{
        big=pic/kartoffeln,     % small picture
        small=pic/bratkartoffeln  % big picture
    }
    
    % \introduction{%
    %     \blindtext
    % }
    
    \ingredients{%
        \unit[2,5]{kg} & Kartoffeln \emph{(festkochend)}\\
        \unit[1,25]{kg} & Saure Sahne\\
        0,5 Knollen & Knoblauch\\
        1 Bund & Petersilie\\
        1 Bund & Schnittlauch\\
        1/4 Tube & Senf\\
        \unit[250]{g} & Zwiebeln
    }
    
    \preparation{%
        \step Die Kartoffeln sauber bürsten \emph{oder} schälen, in Scheiben oder Würfel schneiden (die ganz Verrückten können auch beides machen!) und 5 Minuten in kaltes Wasser einlegen. 
        \step Kartoffeln herausnehmen und beginnen, sie bei kleiner Stufe in Öl anzubraten. Einen Deckel aufzulegen ist sinnvoll, das spart Gas oder Holz. Achtet aber darauf, dass das ganze nicht zu feucht wird, sonst braten die Kartoffeln nicht richtig. Währenddessen können Zwiebeln und Knoblauch zerkleinert werden.
        \step Nach 5 Minuten können die Kartoffeln das erste Mal gewendet werden, Ihr müsst die Kartoffeln regelmäßig kontrollieren, damit sie nicht anbrennen. 
        \step Wenn die Kartoffeln anfangen, eine schön goldbraune Farbe zu bekommen, gebt je die Hälfte von Knoblauch und Zwiebeln hinzu und salzt sie ein wenig. 
        \step Wer zwischendurch Zeit hat und nur in der Küche rumsteht, kann sich die frischen Kräuter schnappen und sie fein hacken und dann gemeinsam mit der anderen Hälfte des Knoblauch-Zwiebel-Gemischs in die Sour Creme einrühren. Eine Idee Senf gibt der Sour Creme den letzten Schliff. 
Danach nur noch salzen und pfeffern und sie ist fertig.
        \step Unter regelmäßigem Wenden solltet Ihr sie so lange braten lassen, bis die Kartoffeln den gewünschten Grad an Knusprigkeit erlangt haben und die Zwiebeln glasig geworden sind. Auch hier noch je eine Prise Salz und Pfeffer (oder auch etwas mehr) hinzugeben, alles servieren und im Essenskreis anschreien. (ÜÜÜÜBEN!!!)
    }
    
    % \hint{%
    %     Enjoy typesetting recipes with {\textbf{\Large\LaTeX}} and {\textbf{\Large xcookybooky!}}
    % }
    
\end{recipe}