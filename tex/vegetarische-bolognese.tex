%!TEX root = ../Kochbuch.tex

% Complete recipe example
\begin{recipe}{vegetarische Bolognese}
    \graph{
        small=pic/parmesan,
        big=pic/bolognese,
    }
    %
    % \introduction{%
    % ``...des mag isch ganz besonners gern...''\flushright --- \emph{Joel "{}böll"{} Fourier}, Friedberg
    % }
    
    \ingredients{%
        500 g & Karotten\\
        0,25 Stück & Knollensellerie\\
        2 Stangen & Lauch\\
        2 kg & Tomaten\\
        0,5 Knollen & Knoblauch\\
        300 g & Zwiebeln\\
        1 Tube & Tomatenmark\\
        1,5 kg & Nudeln\\
        500 g & Zucchini\\
        300 g & Parmesan \emph{(gehobelt)}
    }
    
    \preparation{%
        \step Ärmel hochkrempeln, Vorfreude steigern und los geht’s mit dem Schnibbeln: Alles Gemüse kleinschnibbeln, das ihr habt (neeeeein, die Salatgurke ist vom Mittagessen übrig, die nicht).

        \step Jetzt geht’s richtig los - Kocher an! Lauch, Knollensellerie, Knoblauch und Zwiebel in Öl glasig braten und würzen (Salz, Pfeffer, Paprika, ...).

        \step Karotten, Zucchini und Tomaten zugeben und fröhlich köcheln lassen. Wenn die Karotten langsam weich werden, Tomatenmark einrühren und weiterköcheln auf niedriger Stufe.

        \step Nebenbei Nudelwasser aufsetzen und warten. Jetzt bietet es sich an, im Kochbuch oder Lagerheft zu stöbern.

        \step Wenn das Nudelwasser kocht \emph{(Achtung: ab und an die Soße rühren)}, Salz ins Wasser streuen und die Nudeln al dente kochen – klingt super, gell?

        \step Am besten jetzt schon die hungrigen Mäuler zusammenrufen und die Nudeln abschütten. Nudeln, Soße und Parmesan im Essenskreis servieren – und wenn alle schöne Tomatenmünder haben ein tolles Foto machen: Spaghettiiiiiiii!\
        
    }
    
    \setHeadlines{hinthead = Quizfrage}
    \hint{%
        Welches der beiden Rezepte war zuerst da?
    }
    
\end{recipe}