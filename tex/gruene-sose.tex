%!TEX root = ../Kochbuch.tex

% Complete recipe example
\begin{recipe}{Grüne Sauce mit Kartoffeln}
    \graph{
        small=pic/gruene-sauce,
        big=pic/gruene-kaeuter
    }
    
    \introduction{%
    ``Das ist vergleichsweise einfach, aber man muss viel schneiden.''\flushright --- \emph{Jack the Ripper}
    }
    
    \ingredients{%
        1 Bund & Petersilie\\
        1 Bund & Schnittlauch\\
        0,5 Bund & Pimpinelle\\
        0,5 Bund & Sauerampfer\\
        0,5 Bund & Kerbel\\
        0,5 Bund & Borretsch\\
        1 Pack & Kressesamen\\
        2,5 kg & Kartoffeln\\
        1 kg & Saure Sahne\\
        15 & Eier\\
    }
    
    \preparation{%
        \step Die Kartoffeln können, wie immer, geschält oder nur gewaschen werden. In einem Topf könnt Ihr sie kochen; ihr verbraucht weniger Energie, wenn Ihr den Deckel auf den Topf setzt und ihn nur zu einem Drittel mit gesalzenem Wasser füllt.
        
        \step Die Kräuter, die Ihr jeweils an den Kräuterstationen in Eurer Nähe gesammelt habt, werden kleingeschnitten (sehr fein Hacken gibt der Soße am Ende eine grüne Farbe) und mit der Sauren Sahne vermischt. 
        
Die Eier werden hart gekocht, kleingeschnitten und ebenfalls unter die Soße gemengt.
    }
    
    
    \setHeadlines{hinthead = Fun Fact}
    \hint{%
        Es ist nicht sicher erwiesen, dass Johann Wolfgang von Goethe wirklich dasselbe Rezept für Frankfurter Grüne Soße als sein Leibgericht bezeichnet hat. Was für ein armer Tor.
    }
    
\end{recipe}