%!TEX root = ../Kochbuch.tex

% Complete recipe example
\begin{recipe}{Schinken-Sahne-Sauce mit Nudeln}
    \graph{
        small=pic/schinken-sahne-sauce,
        big=pic/nudeln
    }
    
    % \introduction{%
    % Risotto ist DAS Gericht, das das Prädikat "{}schlotzig"{} trägt. Gemeint ist die Konsistenz, die der Reis erreichen soll und die ist genauso, wie es das Wort vermuten lässt.
    % }
    
    \ingredients{%
       1,3 kg & Nudeln (Vollkorn)\\
       500 g & Schinken\\
       500 g & Zwiebeln\\
       1 Knollen & Knoblauch\\
       700 g & Sahne\\
       400 g & Crème fraîche\\
       1 Bund & Petersilie\\
       2 Bund & Schnittlauch
    }
    
    \preparation{%
        \step Als erstes könnt Ihr die Nudeln nach Anleitung al dente kochen. Wir haben keine Ahnung, was das bedeutet, es klingt aber wahnsinnig gut. Schaut mal nach links, da ist es auch nochmal genau erklärt...
        
        \step Die Zwiebeln und der Knoblauch werden gewürfelt bzw. fein gehackt, gesalzen und in etwas Öl angeschwitzt. 
        
        \step Wenn sie schön glasig sind, dann wird der Schinken, den Ihr vorher in feine Streiflein geschnitten habt, zugegeben. 
        
        \step Danach kann Sahne und Crème Fraîche der Mixtur hinzugefügt werden. Würzt das Ganze mit Salz, Pfeffer und Muskatnuss. 
        
        \step Zum Schluss werden die Petersilie und der (oder das?) Schnittlauch zugeführt. Mit den frisch gekochten Nudeln ergibt sich ein leckeres Hauptgericht. 
        
    }
    %
    % \hint{%
    %     In einigen Stämmen werden ähnliche Gerichte - gegebenenfalls auch ungekocht - unter dem Namen "{}Ingelschleim"{} angepriesen. Aus Gründen der Diskretion gehen wir hier aber nicht näher auf die betroffenen Gruppen ein.
    % }
    
\end{recipe}