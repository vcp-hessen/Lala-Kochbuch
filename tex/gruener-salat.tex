%!TEX root = ../Kochbuch.tex

% Complete recipe example
\begin{recipe}{Grüner Salat mit Joghurtdressing}
    \graph{
        big=pic/gruener-salat,
        small=pic/salatkopf,
    }
    
    % \introduction{%
    % ``Das ist vergleichsweise einfach, aber man muss viel schneiden.''\flushright --- \emph{Jack the Ripper}
    % }
    
    \ingredients{%
        1 Stück & Salatkopf\\
        2 Stück & Paprika\\
        250 g & Tomaten\\
        2 Stück & Gurken\\
        300 g & Joghurt
    }
    
    \preparation{%
        \step Der Salatkopf wird gewaschen und alle braunen Blätter, die sich evtl. außen befinden, werden abgemacht und dem Kompost zugeführt.
        
        \step Dann alle grünen Blätter abrupfen und in mundgerechte Stücke zerpflücken. In einem Wasserbad werden sie noch mal abgewaschen und dann zum Abtropfen aus dem Wasser geholt. 
        
        Wer eine Salatschleuder hat: Benutzen! Wer keine hat, der kann jedes Blatt einzeln trocknen. Oder er nutzt einfach ein Nudelsieb und einen Topfdeckel und schüttelt ausgiebig (aber vorsichtig - sonst müsst ihr wieder waschen). 
        
        \step Mit Gurken, Paprika und Tomaten verfahren wir auf die schon bekannte Weise: Zerkleinern, dann unter den Salat mischen.
        
        \step Das Dressing wird mit Kräutern aus dem nächstgelegenen Kräutergarten, ein wenig Senf und Joghurt zubereitet. Eine Prise Zucker gibt Pfiff. Mit Salz und Pfeffer abschmecken und dann servieren.

    }
    
    
    \setHeadlines{hinthead = Kilian's Trick}
    \hint{%
        Wenn das Mittagessen noch ein bisschen in der Zukunft liegt, dann lasst das Dressing noch aus dem Salat draußen, sonst zieht der Salat zu viel Wasser, verwässert das Dressing und in letzter Konsequenz müssen wir das Landeslager evakuieren. Wenn Ihr alles richtig gemacht habt, freuen wir uns. Lasst es Euch schmecken.
    }
    
\end{recipe}