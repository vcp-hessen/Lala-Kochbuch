%!TEX root = ../Kochbuch.tex

% Complete recipe example
\begin{recipe}{Gemüseeintopf}
    \graph{
        small=pic/gemueseeintopf,
        big=pic/gemueseeintopf2,
    }
    
    % \introduction{%
    %     Sauce? Sose? Sauce? Oder Sauße? Aber Eier!
    % }
    
    \ingredients{%
        0,25 Stück & Knollensellerie\\
        3 Stück & Paprika\\
        2 Stange & Lauch\\
        500 g & Karotten\\
        500 g & Zucchini\\
        400 g & Kartoffeln (mehlig)\\
        200 g & Zwiebeln\\
        1 Tube & Tomatenmark\\
        2 Liter & Gemüsebrühe\\
        1 Laib 1kg & Schwarzbrot
    }
    
    \preparation{%
        \step Den Lauch in Ringe schneiden, Zwiebeln fein und Sellerie grob würfeln. Alles salzen und in einem Topf mit etwas Öl anbraten. 

        \step Die Kartoffeln können geschält, gewürfelt und als nächstes dazugegeben werden. 
Als nächstes sind die Karotten dran und danach kommen Paprika und Zucchini rein. 
        
        \step Wenn alles Gemüse im Topf ist, kann er mit Gemüsebrühe (Pulver und Wasser) aufgefüllt werden. Der Eintopf lässt sich gut mit Tomatenmark verfeinern. Ein guter Kanten Schwarzbrot erlaubt uns, bald satt zu sein. Mjam!

    }
    
    % \hint{%
    %     Wenn Ihr einen Deckel für das Kartoffel-Kochgefäß habt, dann habt Ihr die Möglichkeit, Energie zu sparen: Füllt den Topf nur etwa 1/3 so hoch mit leicht gesalzenem Wasser, wie Kartoffeln drin sind. Tut dann den Deckel drauf, die Kartoffeln werden schneller gar und Ihr spart Gas bzw. Holz! Wer wissen will, wie das funktioniert, fragt Kilian oder Kielius. Die können das erklären!
    % }
    
\end{recipe}