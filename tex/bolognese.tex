%!TEX root = ../Kochbuch.tex

% Complete recipe example
\begin{recipe}{Ragù alla Bolognese}
    \graph{
        big=pic/bolognese-vegi,
        small=pic/tomaten
    }
    %
    % \introduction{%
    % ``...des mag isch ganz besonners gern...''\flushright --- \emph{Joel "{}böll"{} Fourier}, Friedberg
    % }
    
    \ingredients{%
        1 kg & Hackfleisch \emph{(Halb \& Halb)}\\
        1,5 kg & Nudeln\\
        0,5 Knollen & Knoblauch\\
        2 kg & Tomaten\\
        400 g & Zwiebeln\\
        200 g & Karotten\\
        1 Tube & Tomatenmark\\
        300 g & Parmesan \emph{(gehobelt)}
    }
    
    \preparation{%
        \step Ärmel hochkrempeln, Vorfreude steigern und los geht's mit dem Schnibbeln: Zwiebeln, Knoblauch, Karotten und Tomaten in wunderschöne, kleine Stücke schneiden.

        \step Jetzt geht’s richtig los - Kocher an! Hackfleisch mit dem Öl krümelig anbraten. Knoblauch und Zwiebel hinzufügen und weiterbraten, bis Flüssigkeit verdampft ist – Salzen nicht vergessen.

        \step Karotten und Tomaten zugeben und fröhlich köcheln lassen. Nach Bedarf, Wasser nachschütten. Wenn die Karotten langsam weich werden, Tomatenmark einrühren und weiterköcheln auf niedriger Stufe.

        \step Nebenbei Nudelwasser aufsetzen (mit Deckel natürlich) und warten. Jetzt bietet es sich an, im Kochbuch oder Lagerheft zu stöbern...

        \step Wenn das Nudelwasser kocht \emph{(Achtung: ab und an die Soße rühren)}, Salz ins Wasser streuen und die Nudeln al dente kochen – klingt super, gell?
        
        \step Am besten jetzt schon die hungrigen Mäuler zusammenrufen und die Nudeln abschütten. Nudeln, Soße und Parmesan im Essenskreis servieren – und wenn alle schöne Tomatenmünder haben ein tolles Foto machen: Spaghettiiiiiiii!
        
    }
    
    \setHeadlines{hinthead = Anni's Angebot}
    \hint{%
        Die Soße kann ruhig länger köcheln --- das macht sie nur besser. Aber passt auf: Mit dem Chili sparen, lieber mit ein bisschen frischem Parmesan verfeinern. Yummie.
    }
    
\end{recipe}