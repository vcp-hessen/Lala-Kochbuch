%!TEX root = ../Kochbuch.tex

% Complete recipe example
\begin{recipe}{Käsespätzle}
    \graph{
        big=pic/kaesespaetzle,
        small=pic/spaetzle-zutaten
    }
    
    % \introduction{%
    % ``Das ist vergleichsweise einfach, aber man muss viel schneiden.''\flushright --- \emph{Jack the Ripper}
    % }
    
    \ingredients{%
        1,5 kg & Spätzle\\
        400 g & Emmentaler \emph{(gerieben)}\\
        100 g & Gruyere\\
        600 g & Zwiebeln\\
        50 g & Butter\\
        300 g & Crème fraîche
    }
    
    \preparation{%
        \step Die Zwiebeln werden kleingeschnitten und gesalzen. Bei kleiner Flamme werden sie jetzt in Butter angeschwitzt, bis sie schön glasig sind. Wenn sie eine schön flutschige Konsistenz haben, können sie beiseite genommen werden.
        
        \step Jetzt kommen die Spätzle dran: In einem Topf mit gesalzenem Wasser werden die Spätzle nach Anleitung gekocht. Wer noch nicht weiß, wie das geht kann auf Seite 14 blättern, da ist es erklärt...      
        
        \step Wenn die Spätzle fertig sind: abschütten und danach mit dem geriebenen oder klein gehackten Käse in einem großen Topf oder Schüssel vermischt. 
        
        \step Die Zwiebeln werden auch dazugegeben und der Schmand kann entweder so reingemischt oder im Essenskreis zum Selber-Portionieren hingestellt werden.
        
        Käsespätzle vs. Risotto - schwere Wahl...
        
    }
    
    % \hint{%
    %     Energiespartipp: Legt beim Kochen der Kartoffeln den Deckel auf den Topf und füllt diesen nur zu einem Drittel mit Wasser.
    % }
    
\end{recipe}