%!TEX root = ../Kochbuch.tex

% Complete recipe example
\begin{recipe}{Chili sin Carne}
    \graph{
        % small=pic/chili-small,
        small=pic/chili,
        big=pic/chilibohnen
    }
    
    % \introduction{%
    %     \blindtext
    % }
    
    \ingredients{%
        1 kg & Tomaten\\
        3 Stück	& Paprika\\
        600 g & Mais \emph{(Dose)}\\
        400 g & Zwiebeln\\
        0,5 Knollen	& Knoblauch\\
        400 g	& Kidney-Bohnen \emph{(Dose)}\\
        1 Liter	& Gemüsebrühe\\
        1 Tube	& Tomatenmark\\
        400 g	& Karotten\\
        400 g	& Zucchini\\
        800 g	& Reis
    }
    
    \preparation{%
        \step Als erstes könnt Ihr euch um den Reis kümmern, wir empfehlen dabei die \textbf{Quellreis-Methode}: Auf Jeweils einen Teil Reis kommen zwei Teile Wasser. Pro Liter Wasser könnt Ihr einen Teelöffel Salz hinzugeben, dann auf den Kocher stellen und die Kiste anschmeißen. Macht den Deckel drauf, dann geht’s schneller. 
        \step Wenn das Wasser einmal richtig gekocht hat, könnt ihr den Reis vom Kocher nehmen, die restliche Wäre reicht aus um den Reis zu garen. Wenn kein Wasser mehr im Topf ist, ist er fertig - ganz ohne abgießen!
        \step Für das Chili hackt zuerst die Zwiebeln grob und bratet sie in heißem Öl an. Im Grunde schmeißt Ihr jetzt nur nach und nach alle Zutaten dazu und am Ende ist es fertig. Ach, wie herrlich einfach!
        \step In folgender Reihenfolge kommt der Rest rein: Knoblauch, Tomatenmark, Karotten, Zucchini und Paprika, Tomaten, Mais, Gemüsebrühe, Kidneybohnen.
        \step Wenn Ihr wollt, dass das Gemüse noch schön bissfest ist, dann bereitet schon alle Gemüsesorten vor, damit Ihr sie nur noch in den Topf geben müsst. So hat es nicht so viel Zeit, zu verkochen.
        \step Am Ende köchelt das alles vor sich hin und Ihr könnt mal nen Löffel Reis mit Chili probieren. Schmeckt gut? Dann ab in den Essenskreis, wir wünschen guten Hunger!
    }
    
    \setHeadlines{hinthead = Jonas' Tipp}
    \hint{%
        Als Gewürz für Chili sin Carne eignet sich Chili besonders gut! Salz, Pfeffer und Paprikapulver, aber auch italienische Kräuter passt gut. Eine besondere Note bekommt man mit Kreuzkümmel oder mit etwas schwarzer Schokolade.
    }
    
\end{recipe}