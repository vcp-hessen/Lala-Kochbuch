%!TEX root = ../Kochbuch.tex

% Complete recipe example
\begin{recipe}{Eier in Senfsauce}
    \graph{
        small=pic/eier-in-senfsosse,
        big=pic/eier
    }
    
    \introduction{%
        Sauce? Sose? Soße? Oder Sauße? Aber Eier!
    }
    
    \ingredients{%
        20 & Eier\\
        2,5 kg & Kartoffeln\\
        50 g & Butter\\
        100 g & Mehl\\
        200 g & Sahne\\
        0,5 Liter & Milch\\
        2 Glas & Senf \emph{(á 200g)}\\
    }
    
    \preparation{%
        \step Stellt als erstes die geschälten (oder für die fauleren Stämme: gewaschenen) Kartoffeln aufs Feuer. 

        \step Danach könnt Ihr die Eier aufstellen. Wer keinen Anpiekser dabei hat: Gar nicht schlimm. Anstatt irgendwie mit einem Hering ein Loch reinzupulen, lasst sie lieber ganz. Das gefällt den Eiern besser. Etwa 7 Minuten nachdem das Wasser kocht, könnt Ihr die Eier herausnehmen und dann warten sie bestimmt gern auf die anderen Komponenten des Essens.
        
        \step Die [s'oußä] wird folgendermaßen hergerichtet: Erhitzt die Butter in einem Topf und schwitzt darin die klein geschnittenen Zwiebeln an. Wenn die schön glasig sind, fügt langsam und unter stetigem Rühren das Mehl hinzu. Es entsteht eine Mehlschwitze, jetzt schnell (!) Milch und Sahne hinzu und, je nach Geschmack, ein bisschen Gemüsebrühe-Pulver. Wichtig: immer weiter rühren!
        
        \step Die Sauce wird schön dick und kann jetzt mit Senf, Muskatnuss (bitte gerieben) und Salz und Pfeffer abgeschmeckt werden. 
        
        \step Sind die Kartoffeln auch schon fertig? Dann legt das Kochbuch beiseite und nehmt im Essenskreis Platz, bevor es kalt wird. Und Eier nicht vergessen!
    }
    
    
    \setHeadlines{hinthead = Kilian's Trick}
    \hint{%
        Wenn Ihr einen Deckel für das Kartoffel-Kochgefäß habt, dann habt Ihr die Möglichkeit, Energie zu sparen: Füllt den Topf nur etwa 1/3 so hoch mit leicht gesalzenem Wasser, wie Kartoffeln drin sind. Tut dann den Deckel drauf, die Kartoffeln werden schneller gar und Ihr spart Gas bzw. Holz! Wer wissen will, wie das funktioniert, fragt Kilian oder Kielius. Die können das erklären!
    }
    
\end{recipe}