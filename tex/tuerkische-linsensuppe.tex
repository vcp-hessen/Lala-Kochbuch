%!TEX root = ../Kochbuch.tex

% Complete recipe example
\begin{recipe}{Türkische Linsensuppe}
    \graph{
        small=pic/rote-linsen,
        big=pic/tuerkische-linsensuppe
    }
    
    \introduction{%
    ``...des mag isch ganz besonners gern...''\flushright --- \emph{Joel "{}böll"{} Fourier}, Friedberg
    }
    
    \ingredients{%
        1 kg & Linsen (rote)\\
        500 g & Zwiebeln\\
        0,5 Knollen & Knoblauch\\
        500 g & Tomaten\\
        100 g & Butter\\
        5 Liter & Gemüsebrühe\\
        1 Laib 1kg & Schwarzbrot\\
        500 g & Sahne\\
        0,5 Bund & Minze\\
        1 Stück & Zitrone\\
        0,5 Stück & Chilischote
    }
    
    \preparation{%
        \step Zunächst machen wir eine Pfefferminzbutter. Dazu wird Butter geschmolzen und die Minze darin ca. fünf Minuten geschwenkt. Oder geschwunken? Nehmt die Pfefferminzbutter aus dem Topf und gießt sie in einen kleinen Behälter.
        
        \step Jetzt könnt Ihr in Öl die klein geschnittenen Zwiebeln und den Knoblauch anschwitzen. Salzen nicht vergessen. 
        
        \step Als nächstes werft Ihr voll der Achtung für die uns hingegebenen Linsen genau diese in den Topf und füllt ihn mit der Gemüsebrühe auf. 
        
        \step Jetzt köchelt das Gericht vor sich hin, so lange, bis keine Linsen mehr zu erkennen sind. Gebt nun den Saft der Zitronen und etwas kleingehackte Schale der Bio-Zitrone hinzu und lasst es köcheln. 
        
        \step Die Tomaten und die fein gehackte Chili komplettieren das Mahl. Unter Dreingabe der Minzbutter kann die Suppe serviert werden; Dazu passt ein gutes Stück Schwarzbrot!
        
    }

    % \hint{%
    %     Manche Stimmen bezeichnen dieses Gericht als das beste Abendessen, das es heute gibt.
    % }
    
\end{recipe}