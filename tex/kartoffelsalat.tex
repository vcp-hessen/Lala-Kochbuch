%!TEX root = ../Kochbuch.tex

% Complete recipe example
\begin{recipe}{Kartoffelsalat aus Lettland}
    \graph{
        big=pic/kartoffelsalat,
        small=pic/kartoffeln
    }
    
    % \introduction{%
    % ``Das ist vergleichsweise einfach, aber man muss viel schneiden.''\flushright --- \emph{Jack the Ripper}
    % }
    
    \ingredients{%
        4 kg & Kartoffeln \emph{(festkochend)}\\
        10 Stück & Eier\\
        1 kg & Saure Sahne\\
        500 g & Saure Gurken\\
        300 g & Crème fraîche\\
        2 Bund & Dill\\
        200 g & Frühlingszwiebeln\\
        etwas & Senf
    }
    
    \preparation{%
        \step Die Kartoffeln können als erstes geschält und in Würfel oder Scheiben geschnitten werden. Kocht sie ca. 15 Minuten lang in gesalzenem Wasser und gießt sie ab, wenn sie gar sind.
        
        Das solltet ihr ruhig auch schon lange vor dem Essen machen - soll ja kalt werden.
        
        \step Verfahrt ebenso mit den Eiern, aber schneidet sie erst nach dem Kochen in Würfel :-) Die Eier brauchen ca. 7 Minuten. Aber lieber etwas zu lang als zu kurz.        
        
        \step Die klein geschnittenen Gurken mit Saurer Sahne, Dill (gehackt), klein geschnittenen Frühlingszwiebeln, Eiern und Kartoffeln zusammenmischen. Gebt noch einen ordentlichen Schluck von dem Gurkenwasser dazu, das gibt dem Salat eine feinsaure Note.
        
        \step Mit Senf, Salz und Pfeffer abschmecken und Ihr habt ein köstliches Stück Baltikum zubereitet. Lasst es Euch schmecken!
        
    }
    
    \setHeadlines{hinthead = *pssst* Küchengeheimnis}
    \hint{%
        Energiespartipp: Legt beim Kochen der Kartoffeln den Deckel auf den Topf und füllt diesen nur zu einem Drittel mit Wasser.
    }
    
\end{recipe}