\documentclass{article}
\usepackage{xparse}
\usepackage{fontspec}
\setmainfont{Bradley Hand}
\usepackage{tikz}
\usetikzlibrary{shadows}
\usepackage{lipsum}

\definecolor{myyellow}{RGB}{242,226,149}

\NewDocumentCommand\StickyNote{O{6cm}mO{6cm}}{%
\begin{tikzpicture}
\node[
drop shadow={
  shadow xshift=2pt,
  shadow yshift=-4pt
},
inner xsep=7pt,
fill=myyellow,
xslant=-0.1,
yslant=0.1,
inner ysep=10pt
] {\parbox[t][#1][c]{#3}{#2}};
\end{tikzpicture}%
}

\begin{document}
\pagenumbering{gobble}

\StickyNote{\large Unsere Kräuter gibts übrigens frisch aus dem Kräutertopfgarten. Schnappt euch das, was ihr braucht, aber hey - lasst die Pflanze ganz.\\~\\Und ganz wichtig: gießen nicht vergessen! Schaut einfach, wer den grünsten Daumen hat - vielleicht kann er oder sie Kräuterchef werden.}

% \StickyNote[2.5cm]{\LARGE As \underline{who}, I want\\[4ex]\underline{what} so that \underline{why}}[6.5cm]

\end{document}